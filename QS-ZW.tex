%% March 2018
%%%%%%%%%%%%%%%%%%%%%%%%%%%%%%%%%%%%%%%%%%%%%%%%%%%%%%%%%%%%%%%%%%%%%%%%%%%%
% AGUJournalTemplate.tex: this template file is for articles formatted with LaTeX
%
% This file includes commands and instructions
% given in the order necessary to produce a final output that will
% satisfy AGU requirements, including customized APA reference formatting.
%
% You may copy this file and give it your
% article name, and enter your text.
%
%
% Step 1: Set the \documentclass
%
% There are two options for article format:
%
% PLEASE USE THE DRAFT OPTION TO SUBMIT YOUR PAPERS.
% The draft option produces double spaced output.
%

%% To submit your paper:
\documentclass[draft,linenumbers]{agujournal2018}
\usepackage{apacite}
\usepackage{url} %this package should fix any errors with URLs in refs.
%%%%%%%
% As of 2018 we recommend use of the TrackChanges package to mark revisions.
% The trackchanges package adds five new LaTeX commands:
%
%  \note[editor]{The note}
%  \annote[editor]{Text to annotate}{The note}
%  \add[editor]{Text to add}
%  \remove[editor]{Text to remove}
%  \change[editor]{Text to remove}{Text to add}
%
% complete documentation is here: http://trackchanges.sourceforge.net/
%%%%%%%


%% Enter journal name below.
%% Choose from this list of Journals:
%
% JGR: Atmospheres
% JGR: Biogeosciences
% JGR: Earth Surface
% JGR: Oceans
% JGR: Planets
% JGR: Solid Earth
% JGR: Space Physics
% Global Biogeochemical Cycles
% Geophysical Research Letters
% Paleoceanography and Paleoclimatology
% Radio Science
% Reviews of Geophysics
% Tectonics
% Space Weather
% Water Resources Research
% Geochemistry, Geophysics, Geosystems
% Journal of Advances in Modeling Earth Systems (JAMES)
% Earth's Future
% Earth and Space Science
% Geohealth
%
% ie, \journalname{Water Resources Research}

\journalname{Geophysical Research Letters}

\usepackage{gensymb}
\usepackage{soulutf8}
\usepackage{subfig}

\begin{document}

%% ------------------------------------------------------------------------ %%
%  Title
%
% (A title should be specific, informative, and brief. Use
% abbreviations only if they are defined in the abstract. Titles that
% start with general keywords then specific terms are optimized in
% searches)
%
%% ------------------------------------------------------------------------ %%

% Example: \title{This is a test title}

\title{How stationary are planetary waves in the Southern Hemisphere?}

%% ------------------------------------------------------------------------ %%
%
%  AUTHORS AND AFFILIATIONS
%
%% ------------------------------------------------------------------------ %%

% Authors are individuals who have significantly contributed to the
% research and preparation of the article. Group authors are allowed, if
% each author in the group is separately identified in an appendix.)

% List authors by first name or initial followed by last name and
% separated by commas. Use \affil{} to number affiliations, and
% \thanks{} for author notes.
% Additional author notes should be indicated with \thanks{} (for
% example, for current addresses).

% Example: \authors{A. B. Author\affil{1}\thanks{Current address, Antartica}, B. C. Author\affil{2,3}, and D. E.
% Author\affil{3,4}\thanks{Also funded by Monsanto.}}

\authors{
Elio Campitelli
\affil{1}
Carolina Vera
\affil{1, 2}
Leandro Díaz
\affil{1, 2}
}


% \affiliation{1}{First Affiliation}
% \affiliation{2}{Second Affiliation}
% \affiliation{3}{Third Affiliation}
% \affiliation{4}{Fourth Affiliation}

\affiliation{1}{Centro de Investigaciones del Mar y la Atmosfera, UMI-IFAECI
(CONICET-UBA-CNRS)}
\affiliation{2}{Departamento de Ciencias de la Atmósfera y los Océanos (FCEyN, UBA)}
%(repeat as many times as is necessary)

%% Corresponding Author:
% Corresponding author mailing address and e-mail address:

% (include name and email addresses of the corresponding author.  More
% than one corresponding author is allowed in this LaTeX file and for
% publication; but only one corresponding author is allowed in our
% editorial system.)

% Example: \correspondingauthor{First and Last Name}{email@address.edu}
\correspondingauthor{Elio Campitelli}{elio.campitelli@cima.fcen.uba.ar}

%% Keypoints, final entry on title page.

%  List up to three key points (at least one is required)
%  Key Points summarize the main points and conclusions of the article
%  Each must be 100 characters or less with no special characters or punctuation

% Example:
% \begin{keypoints}
% \item	List up to three key points (at least one is required)
% \item	Key Points summarize the main points and conclusions of the article
% \item	Each must be 100 characters or less with no special characters or punctuation
% \end{keypoints}

\begin{keypoints}
\item Zonal waves and Quasi-stationary waves are disctinct but related
phenomena
\item This distinction has theoretical and practical implications
\item The relationship between the mean ZW amplitude and QS amplitude yields
an estimate of stationarity
\end{keypoints}

%% ------------------------------------------------------------------------ %%
%
%  ABSTRACT
%
% A good abstract will begin with a short description of the problem
% being addressed, briefly describe the new data or analyses, then
% briefly states the main conclusion(s) and how they are supported and
% uncertainties.
%% ------------------------------------------------------------------------ %%

%% \begin{abstract} starts the second page

\begin{abstract}
Abstract goes here
\end{abstract}

\section{Introduction}

Even though many atmospheric variables depend strongly on latitude, they
can deviate considerably from their latitudinally averaged value. These
deviations modulate weather systems and regional climate through
latitudan transport of heat, humidity, and momentum (REFS), and also can
contribute to blocking patterns (REF).

Because of their wave-like structure, \citet{Loon1972} called the zonal
asymmetries in the time-mean southern hemisphere geopotential ``standing
waves'' and distinguished them from the asymmetries of the daily fields,
which he called ``daily waves''. Other studies use different
terminology: \citet{Quintanar1995a} and \citet{Rao2004} called them
``quasi-stationary waves'', \citet{Turner2017} used ``planetary waves''
as a synonym, and \citet{Raphael2004} and \citet{Irving2015} called them
``zonal waves''. \citet{Kravchenko2012} and \citet{Lastovicka2018} used
the terms ``quasi-stationary waves'' and ``stationary planetary waves'',
respectively, but in reference to waves in the individual fields (the
``daily waves'', following \citet{Loon1972} terminology).

These studies also use different methods. \citet{Loon1972} and
\citet{Quintanar1995a} averaged the fields and then computed the wave
amplitude, while \citet{Rao2004} and \citet{Turner2017} computed the
wave amplitudes of the individual fields and then averaged the
amplitudes. \citet{Raphael2004} and \citet{Irving2015} constructed
indexes of the amplitude of planetary waves, but the former one is
sensitive to waves in phase with the mean wave, while the latter
captures all waviness irrespective of phase.

Quasi-stationary waves arise from the superposition of individual
``daily waves'' with similar phase. However, no recent studies assessed
\emph{how} similar, and thus, how ``stationary'' are
``quasi-stationary'' waves.

In this study we aim to distinguish between quasi-stationary waves and
zonal waves and to show that the mean amplitude of zonal waves differ
from the amplitude of quasi-stationary waves. We exploit this difference
to construct a measure of quasi-stationary wave stationarity and show
that planetary waves with wavenumbers 2 and 3 are significantly less
stationary in the southern hemisphere than in the northern hemisphere.

\section{Methods}

We define \emph{planetary waves} as waves that extend along a full
latitude circle. \emph{Zonal waves} (ZW) are planetary waves of the
``instantaneous'' fields and \emph{quasi-stationary waves} (QS),
planetary waves of the time-mean field such that:

\begin{linenomath*}
\begin{eqnarray}\label{eq:ZW}
\mathrm{ZWk}(t) & = & A_\mathrm{ZWk}(t)\cos \left [ k\lambda - \alpha_\mathrm{ZWk}(t) \right ] \\ 
\overline{\mathrm{ZWk}(t)} = \mathrm{QSk} & = & A_\mathrm{QSk}\cos \left (  \mathrm{k}\lambda - \alpha_\mathrm{QSk} \right ) \label{eq:QS}
\end{eqnarray}
\end{linenomath*}

where \(\mathrm{k}\) is wavenumber, \(\lambda\) longitude, and
\(\mathrm{A_{x}}\) and \(\alpha_\mathrm{x}\), amplitude and phase,
respectively. \(\mathrm{ZWk}(t)\) depends on time, but not
\(\mathrm{QSk}\). From the properties of wave superposition we can
deduce that, in general, the stationary phase \(\alpha_\mathrm{QSk}\)
does not equal \(\overline{\alpha_\mathrm{ZWk}}\) and the stationary
amplitude \(A_\mathrm{QSk}\) is less or equal
\(\overline{A_\mathrm{ZWk}}\) \citep{Pain2005}.

These definitions depend on which are the ``instantaneous fields'' and
the averaging time-scales. A dataset of 365 daily mean fields defines
365 daily zonal waves and one annual quasi-stationary wave but 12
monthly quasi-stationary waves (per level and latitude). A 30 year
dataset of monthly mean fields define 360 monthly zonal waves and one
30-year quasi-stationary wave. Monthly planetary waves are
quasi-stationary waves in one case and zonal waves in the other.

Here we use monthly geopotential fields from the NCEP/NCAR Reanalysis
\citep{Kalnay1996} for the period 1948 to 2017 and compute one
quasi-stationary wave for the whole period for each month, level and
wavenumber. Amplitude and phase for each wavenumber was estimated by
fitting a fourier transform for each latitude circle, level and monthly
record.

\section{Results}

\begin{figure}[h]

{\centering \subfloat[At 50\degree N\label{fig:rao1}]{\includegraphics{fig/QS-ZW/rao-1} }\newline\subfloat[At 50\degree S\label{fig:rao2}]{\includegraphics{fig/QS-ZW/rao-2} }

}

\caption{Seasonal cycle of amplitude of the geopotential planetary waves 1 to 3 computed as the amplitude of the mean wave ($A_\mathrm{QSk}$) and as the mean amplitude of the monthly waves ($\overline{A_\mathrm{ZW} }$).}\label{fig:rao}
\end{figure}

Figure \ref{fig:rao} shows the seasonal cycle of the amplitude of
planetary waves at 50\degree S and 50\degree N using monthly fields from
the NCEP/NCAR reanalysis \citep{Kalnay1996} between 1948 and 2017. We
computed the left column (\(A_\mathrm{QS}\)) as the amplitude of the
average geopotential field for each month, level and wavenumber, and the
right column (\(\overline{A_\mathrm{ZW}}\)) as the average amplitude of
the 70 individual fields.

Figure \ref{fig:rao1} shows that at 50\degree N for the three
wavenumbers \(A_\mathrm{QS}\) and \(\overline{A_\mathrm{ZW}}\) have a
similar seasonal cycle with similar vertical extent. In the southern
hemisphere, however, this is true only for wavenumber 1 (Figure
\ref{fig:rao2}). \(A_\mathrm{QS2}\) is much smaller than
\(\overline{A_\mathrm{ZW2}}\) and its seasonal cycle is less defined.
\(A_\mathrm{QS3}\) has a smaller magnitude than
\(\overline{A_\mathrm{ZW3}}\) end even though their overall structure is
similar (one relative maximum in February-March in the middle
troposphere and another in July-August that extends to the lower
stratosphere), they differ in the details. \(A_\mathrm{QS3}\) has a
local minimum in November that is absent in
\(\overline{A_\mathrm{ZW3}}\). The relative contribution of each
wavenumber is also different. While \(\overline{A_\mathrm{ZW2}}\)
dominates over \(\overline{A_\mathrm{ZW3}}\) in the stratosphere and is
of similar magnitude in the troposphere, \(A_\mathrm{QS3}\) dominates
over \(A_\mathrm{QS2}\) throughout the year and in every level except in
the aforementioned November minimum.

\citet{Loon1972} also recognized these differences. He observed that
daily zonal waves 2, 4, 5 and 6 had big amplitudes but, unlike zonal
waves 1 and 3, their quasi-stationary wave counterparts were negligible.
He deduced that zonal waves 1 and 3 were exceptionally consistent in
phase and thus had what he called a ``standing wave component''. We
quantify this observation as the quotient between \(A_\mathrm{QS}\) and
\(\overline{A_\mathrm{ZW}}\). As an analogy with the constancy of the
wind \citep{Singer1967}, we define quasi-stationary wave stationarity as

\begin{linenomath*}
\begin{equation}\label{eq:S}
\hat{S} = \frac{A_\mathrm{QS}}{\overline{A_\mathrm{ZW}}}
\end{equation}
\end{linenomath*}

For a sample of \(n\) completely random waves, the expected value of
\(\hat{S}\) is \(n^{-1/2}\) because the average amplitude of the sum of
\(n\) waves with random phases and mean amplitude \(A\) is \(An^{-1/2}\)
\citep{Pain2005}. For completely stationary waves \(\hat{S} = 1\)
irrespective of sample size.

While \(\hat{S}\) is used --sometimes as
\(2/\pi\arcsin \left (\hat{S} \right )\) \citep{Singer1967}-- in the
meteorological literature in the context of wind steadiness
\citep[e.g][]{Hiscox2010}, to our knowledge this is the first time it
has been applied to the study of atmospheric waves.

\begin{figure}[h]

{\centering \includegraphics{fig/QS-ZW/stationarity-1} 

}

\caption{Seasonal cycle of stationarity at 50\degree S and 50\degree N computed using Equation \ref{eq:S}}\label{fig:stationarity}
\end{figure}

Figure \ref{fig:stationarity} shows \(\hat{S}\) for wavenumbers 1 to 3
computed using Equation \ref{eq:S} at 50\degree N and 50\degree S. We
separate between high and low stationarity with the ad-hoc threshold of
0.4 (black line in Figure \ref{fig:stationarity}).

At 50\degree N planetary waves 1, 2 and 3 are highly stationary in
almost every month and level, and even more so planetary wave 1 at
50\degree S.

In the southern hemisphere, planetary wave 2 stationarity has a
semianual cycle. It reaches its maximum in April and in
August-September, plummeting to a deep minimum in June. Planetary wave 3
stationarity peaks in February and slowly decreases towards a November
deep minimum after witch increases sharply.

Equation \ref{eq:S} is equivalent to

\begin{linenomath*}
\begin{equation}\label{eq:S2}
\hat{S} =   \frac{\sum_t A_\mathrm{ZW}(t) \cos  \left [\alpha_\mathrm{zw}(t) - \alpha_{qs} \right ]}{\sum_t A_\mathrm{ZW}(t)}
\end{equation}
\end{linenomath*}

The numerator represents the sum of the zonal waves amplitudes projected
onto the direction of the quasi-stationary wave. Waves that deviate from
that direction decrease stationarity in proportion to their amplitude.

We used Equation \ref{eq:S2} to compute a timeseries of quasi-stationary
wave stationarity. We first calculated \(\alpha_{qs}\) for each month
and then, applied Equation \ref{eq:S2} with a 15-year rolling window
approximated using loess regression with degree 0. The results for
wavenumbers 1 to 3 at 50\degree N and 50\degree S are shown in Figure
\ref{fig:s-timeseries}.

\begin{figure}[h]

{\centering \includegraphics{fig/QS-ZW/s-timeseries-1} 

}

\caption{Quasi-stationary wave stationarity for wavenumbers 1 to 3}\label{fig:s-timeseries}
\end{figure}

Quasi-stationary wave stationarity remained high and constant for
wavenumbers 1 to 3 at 50\degree N and 1 at 50\degree S but not for
wavenumbers 2 and 3 at 50\degree S. Quasi-sationary wave 3 stationarity
jumped from zero to more than 0.5 in less than five years in the 50's
and increased again in the late 70's. These could indicate
inhomogeneities caused by changes in the observational network --routine
satellite observations began in 1979-- but the absense of similar breaks
for wavenumbers 1 or 2 suggest they represent real changes in the
atmospheric circulation with unknown cause.

\subsection{Considerations about phase}

\label{sec:phase}

For defining local impacts, the phase of planetary waves is as important
as their amplitude. One way of dealing with the phase of ZW is to fix
it. \citet{Yuan2008} use Principal Component Analysis on the meridional
wind field to obtain a spatial pattern of the leading mode that is very
similar to the QS3. The timeseries associated to this mode is, then, an
indication of the intensity of the ZW3 that is similar to the QS3. A
more direct approach is the index created by \citet{Raphael2004}. Since
it is based on the geopotential height anomalies at the maximums of the
QS3, it is sensitive to ZW3 patterns with phase close to the stationary
phase. An almost mathematically equivalent approach (with correlation =
0.98) is to compute the projection of each \(\mathrm{ZW}\) onto the
direction of the \(\mathrm{QS}\) (i.e.~the expression inside the sum of
the numerator in Equation \ref{eq:S2}). This methodology has fewer
constrains in that the phase of interest can be changed depending on the
application.

\section{Conclusions}

The fact that zonal waves (ZW) and quasi-stationary waves (QS) are two
distinct but related phenomena has both practical and theoretical
implications.

First, researchers should be aware of which phenomena they want to study
and use the appropriate methods. The mean amplitude of zonal waves could
be appropriate to study the vertical propagation of Rossby waves, for
example. But zonal wave amplitude could lead to misleading results if
used as the basis of local impacts studies because they are probably
more influenced by phase effects.

Secondly, comparison between results should also be made having this
issues in mind. For instance, \citet{Irving2015} compare their planetary
wave activity index with \citet{Raphael2004}'s wave 3 index and conclude
that the later cannot account for events with waves far removed from
their climatological position. However, being an index of the zonal wave
component in phase with the quasi-stationary wave, this is by design.

Although having a consistent nomenclature across papers is important, we
believe this problem can be ameliorated by researchers detailing their
definitions and methodology. This is also good for clarity and
reproducibility. Since planetary waves are generally more stationary in
the northern hemisphere, these issues are more critical for studies of
the southern hemisphere.

Thirdly, the explorations of both ZW and QS can lead to novel levels of
analysis. Here, we showed it can be used to define a metric of
stationarity of quasi-stationary waves, but other applications are also
possible. \citet{Smith2012} used the phase relationship between ZW1 and
QS1 to show that linear interference between the QS1 and ZW1 was related
to vertical wave activity transport at the tropopause.

\begin{center}\rule{0.5\linewidth}{\linethickness}\end{center}

We speculate that the level of stationarity responds to the nature of
the forcings. Higly stationary planetary waves are explained mainly by
stationary forcings while low stationary planetary waves respond mainly
to the internal variation of the atmosphere. This suggests that in the
southern midlatitudes, wave 2 and 3 consist of forced responses mixed
with internat variability. Their annual cycle further suggests that the
mean state of the atmosphere can modulate these responses.

\emph{xx me falta un final acá xx}

\bibliography{qszw.bib}


\end{document}
