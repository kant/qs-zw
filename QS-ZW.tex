%% March 2018
%%%%%%%%%%%%%%%%%%%%%%%%%%%%%%%%%%%%%%%%%%%%%%%%%%%%%%%%%%%%%%%%%%%%%%%%%%%%
% AGUJournalTemplate.tex: this template file is for articles formatted with LaTeX
%
% This file includes commands and instructions
% given in the order necessary to produce a final output that will
% satisfy AGU requirements, including customized APA reference formatting.
%
% You may copy this file and give it your
% article name, and enter your text.
%
%
% Step 1: Set the \documentclass
%
% There are two options for article format:
%
% PLEASE USE THE DRAFT OPTION TO SUBMIT YOUR PAPERS.
% The draft option produces double spaced output.
%

%% To submit your paper:
\documentclass[draft,linenumbers]{agujournal2018}
\usepackage{apacite}
\usepackage{url} %this package should fix any errors with URLs in refs.
%%%%%%%
% As of 2018 we recommend use of the TrackChanges package to mark revisions.
% The trackchanges package adds five new LaTeX commands:
%
%  \note[editor]{The note}
%  \annote[editor]{Text to annotate}{The note}
%  \add[editor]{Text to add}
%  \remove[editor]{Text to remove}
%  \change[editor]{Text to remove}{Text to add}
%
% complete documentation is here: http://trackchanges.sourceforge.net/
%%%%%%%


%% Enter journal name below.
%% Choose from this list of Journals:
%
% JGR: Atmospheres
% JGR: Biogeosciences
% JGR: Earth Surface
% JGR: Oceans
% JGR: Planets
% JGR: Solid Earth
% JGR: Space Physics
% Global Biogeochemical Cycles
% Geophysical Research Letters
% Paleoceanography and Paleoclimatology
% Radio Science
% Reviews of Geophysics
% Tectonics
% Space Weather
% Water Resources Research
% Geochemistry, Geophysics, Geosystems
% Journal of Advances in Modeling Earth Systems (JAMES)
% Earth's Future
% Earth and Space Science
% Geohealth
%
% ie, \journalname{Water Resources Research}

\journalname{Geophysical Research Letters}

\usepackage{gensymb}
\usepackage{soulutf8}
\usepackage{subfig}

\begin{document}

%% ------------------------------------------------------------------------ %%
%  Title
%
% (A title should be specific, informative, and brief. Use
% abbreviations only if they are defined in the abstract. Titles that
% start with general keywords then specific terms are optimized in
% searches)
%
%% ------------------------------------------------------------------------ %%

% Example: \title{This is a test title}

\title{How stationary are planetary waves in the Southern Hemisphere?}

%% ------------------------------------------------------------------------ %%
%
%  AUTHORS AND AFFILIATIONS
%
%% ------------------------------------------------------------------------ %%

% Authors are individuals who have significantly contributed to the
% research and preparation of the article. Group authors are allowed, if
% each author in the group is separately identified in an appendix.)

% List authors by first name or initial followed by last name and
% separated by commas. Use \affil{} to number affiliations, and
% \thanks{} for author notes.
% Additional author notes should be indicated with \thanks{} (for
% example, for current addresses).

% Example: \authors{A. B. Author\affil{1}\thanks{Current address, Antartica}, B. C. Author\affil{2,3}, and D. E.
% Author\affil{3,4}\thanks{Also funded by Monsanto.}}

\authors{
Elio Campitelli
\affil{1}
Carolina Vera
\affil{1, 2}
Leandro Díaz
\affil{1, 2}
}


% \affiliation{1}{First Affiliation}
% \affiliation{2}{Second Affiliation}
% \affiliation{3}{Third Affiliation}
% \affiliation{4}{Fourth Affiliation}

\affiliation{1}{Centro de Investigaciones del Mar y la Atmosfera, UMI-IFAECI
(CONICET-UBA-CNRS)}
\affiliation{2}{Departamento de Ciencias de la Atmósfera y los Océanos (FCEyN, UBA)}
%(repeat as many times as is necessary)

%% Corresponding Author:
% Corresponding author mailing address and e-mail address:

% (include name and email addresses of the corresponding author.  More
% than one corresponding author is allowed in this LaTeX file and for
% publication; but only one corresponding author is allowed in our
% editorial system.)

% Example: \correspondingauthor{First and Last Name}{email@address.edu}
\correspondingauthor{Elio Campitelli}{elio.campitelli@cima.fcen.uba.ar}

%% Keypoints, final entry on title page.

%  List up to three key points (at least one is required)
%  Key Points summarize the main points and conclusions of the article
%  Each must be 100 characters or less with no special characters or punctuation

% Example:
% \begin{keypoints}
% \item	List up to three key points (at least one is required)
% \item	Key Points summarize the main points and conclusions of the article
% \item	Each must be 100 characters or less with no special characters or punctuation
% \end{keypoints}

\begin{keypoints}
\item We devised a quantitative measure of planetary wave stationarity
\item In the southern hemisphere, waves 2 and 3 have interseasonal and decadal
variations in stationarity
\end{keypoints}

%% ------------------------------------------------------------------------ %%
%
%  ABSTRACT
%
% A good abstract will begin with a short description of the problem
% being addressed, briefly describe the new data or analyses, then
% briefly states the main conclusion(s) and how they are supported and
% uncertainties.
%% ------------------------------------------------------------------------ %%

%% \begin{abstract} starts the second page

\begin{abstract}
Many studies of quasi-stationary planetary waves in the Southern
Hemisphere (SH) assumed their quasi-stationary nature based on
\citet{vanloon1972}. However, that study considered only 2 years of data
(1957-1958) before the advent of reanalysis datasets. In this study, we
assessed the stationary conditions in the SH and contrasted it with that
of the Northern Hemisphere using the NCEP/NCAR reanalysis from 1948 to
2017. We also devised a quantitative measure of planetary-wave
stationarity. We confirm that in the SH planetary wave 1 is highly
stationary. Planetary waves 2 and 3 have a comparable mixture of
stationary and moving components with significant variability on
interseasonal and decadal timescales. A deeper knowledge of those
variations could help to better understand the the response of the
mid-latitude atmospheric circulation to surface forcings, caused either
by their strength, or the sensitivity of the atmosphere to them.
\end{abstract}
\noindent{\bf Plain language summary}\vskip-\parskip

\noindent{Large-scale waves in the atmosphere can have stationary and travelling
components. Many studies that focus on the stationary part assume their
stationary nature based on the results of \citet{vanloon1972}, which
studied only 2 years of data (1957-1958) before the existence of modern
reanalysis datasets. In this study we evaluated the stationarity nature
of large-scale waves in the Southern Hemisphere in contrast with the
Northern Hemisphere using the NCEP/NCAR reanalysis from 1948 to 2017. We
also created a quantitative measure of wave stationarity. The results
show that, in the Southern Hemisphere, waves with one maximum per
latitude circle are highly stationary during the whole period and
throughout the year. Higher frequency waves (2 or 3 maximums) have a
comparable mix of stationary and travelling components that varies in
between each season and between decades. These variations could mirror
either variations in external factors or the sensitivity of the
atmosphere to them.}
\vskip18pt
\section{Introduction}

Zonal asymmetries of extratropical circulation in the Southern
Hemisphere (hereafter called as ``planetary waves'') strongly modulate
weather systems and regional climate through latitudinal transport of
heat, humidity, and momentum (\citet{trenberth1980a}). They may also
contribute to the development of blocking events
\citep[e.g.~][]{trenberth1985}, for example, via quasi-resonant
amplifications \citep{petoukhov2013}.

In Rossby wave theory, stationary waves are those with zero frequency or
phase velocity \citep{holton2012}. In practice, however, most studies
focusing on planetary waves in the Southern Hemisphere (HS) assumed
their quasi-stationary nature based on \citet{vanloon1972}. In this
foundational study, the authors analyzed data only from two years, from
1957 and 1958 and found that while extratropical waves with wavenumber 1
to 6 had comparable amplitudes in daily fields, only wavenumbers 1 and 3
featured significantly in the climatological field. From that, they
concluded that only waves 1 and 3 recur consistently in the same
location and thus have a significant quasi-stationary component on top
of a ``moving'' component. This was a qualitative conclusion and to our
knowledge no further study has actually quantified the level of
stationarity associated to each wavenumber.

Quantifying planetary wave stationarity in the \citet{vanloon1972} sense
can be important both in forcasting and process understanging. On one
hand, stationary planetary waves mihgt be more predictable than their
non-stationary counterpart and on the other, the degree of stationarity
is probably an indication of the sensitivity to external forcings.
Moreover, it has also methodological implications. For example,
interpretation of correlations and regressions between planetary wave
amplitudes and other atmospheric and surface variables
\citep[e.g.~][]{turner2017} are only straightforward in the case of
highly stationary planetary waves.

After more than four decades from the publication of
\citet{vanloon1972}, and considering the current availability of
different global reanalysis datasets, in this study we assess the
stationarity features of planetary waves in the SH. Also, we extend
\citet{vanloon1972}'s study, deriving a methodology that provides a
quantitative measure of planetary wave stationarity. We apply it to both
hemispheres.

\section{Methods}

\subsection{Planetary waves}

We define \emph{planetary waves} as waves that extend along a full
latitude circle. \emph{Zonal waves} (ZW) are planetary waves of the
``instantaneous'' fields and \emph{quasi-stationary waves} (QS),
planetary waves of the time-mean field such that:

\begin{linenomath*}
\begin{eqnarray}\label{eq:ZW}
\mathrm{ZWk}(t) & = & A_\mathrm{ZWk}(t)\cos \left [ k\lambda - \alpha_\mathrm{ZWk}(t) \right ] \\ 
\overline{\mathrm{ZWk}(t)} = \mathrm{QSk} & = & A_\mathrm{QSk}\cos \left (  \mathrm{k}\lambda - \alpha_\mathrm{QSk} \right ) \label{eq:QS}
\end{eqnarray}
\end{linenomath*}

where \(\mathrm{k}\) is wavenumber, \(\lambda\) longitude, and
\(\mathrm{A_{ZWk}}\), \(\alpha_\mathrm{ZWk}\) \(\mathrm{A_{QSk}}\) and
\(\alpha_\mathrm{QSk}\), are the amplitudes and phases of each wave.
\(\mathrm{ZWk}(t)\) depends on time, but not \(\mathrm{QSk}\).
Quasi-strationary waves could be more correctly called ``climatological
planetary waves'', but we chose this name for consistency with some of
the previous literature \citep[e.g.~][]{quintanar1995, turner2017}.
While these definitions can, in principle, be applied to any wavenumber,
only long waves are considered true planetary waves and thus here we
limit our analysis to weavenumbers 1 to 3.

These definitions depend on which are the ``instantaneous fields'' and
the averaging time-scales. For example, a dataset of 365 daily mean
fields defines 365 daily zonal waves and one annual quasi-stationary
wave as well as 12 monthly quasi-stationary waves (per level and
latitude). On the other hand, a 30-year dataset of monthly mean fields
defines 360 monthly zonal waves and one 30-year quasi-stationary wave.
While monthly planetary waves are quasi-stationary waves in the first
case, they are zonal waves in the second. The latter shows that the
definition of quasi-stationary waves depends on the temporal sampling
considered.

\subsection{Stationarity}

From the properties of the superposition of waves we can deduce that, in
general, the stationary phase \(\alpha_\mathrm{QSk}\) does not equal
\(\overline{\alpha_\mathrm{ZWk}}\), and that the stationary amplitude
\(A_\mathrm{QSk}\) is less or equal to \(\overline{A_\mathrm{ZWk}}\)
\citep{pain2005}. We use this latter property and use the quotient
between \(A_\mathrm{QSk}\) and \(\overline{A_\mathrm{ZWk}}\) to define,
for each wavenumber \(k\), a measure of quasi-stationary wave
stationarity:

\begin{linenomath*}
\begin{equation}\label{eq:S}
\hat{S_k} = \frac{A_\mathrm{QSk}}{\overline{A_\mathrm{ZWk}}}
\end{equation}
\end{linenomath*}

For a sample of \(n\) completely random waves, the expected value of
\(\hat{S}\) is \(n^{-1/2}\) because the average amplitude of the sum of
\(n\) waves with random phases and mean amplitude \(A\) is \(An^{-1/2}\)
\citep{pain2005}. For completely stationary waves, \(\hat{S}\) is equal
to 1 regardless of sample size.

Some studies consider \(\hat{S}\) as
\(2/\pi\arcsin \left (\hat{S} \right )\) \citep{singer1967} for analyzin
wind steadiness \citep[e.g][]{hiscox2010}. To our knowledge this is the
first time that this approach is applied to study atmospheric waves.

\(\hat{S}\) could be equivalent formulated as

\begin{linenomath*}
\begin{equation}\label{eq:S2}
\hat{S_k} =   \frac{\sum_t A_\mathrm{ZWk}(t) \cos  \left [\alpha_\mathrm{ZWk}(t) - \alpha_{QSk} \right ]}{\sum_t A_\mathrm{ZWk}(t)}
\end{equation}
\end{linenomath*}

The numerator represents the sum of the zonal waves amplitudes projected
onto the direction of the quasi-stationary wave. Waves that deviate from
that direction decrease the overall stationarity in proportion to their
amplitude. This definition of stationarity depends on the phase
distribution and its relationship with amplitude. As it does not depend
on the propagating properties of waves, it's a statistical --rather than
dynamical-- property.

We used Equation \ref{eq:S2} to compute a timeseries of quasi-stationary
wave stationarity. We first calculated \(\alpha_{QSk}\) for each month
and then applied Equation \ref{eq:S2} with a 15-year rolling window
approximated using loess regression with degree 0.

A variety of indices that are used in the literature to study zonal
waves / planetary waves. \citet{raphael2004} explicitly defined a
``zonal wave 3 index'' based on averaging the (temporal) anomalies of
the three-monthly geopotential height mean at three points that roughly
coincide with the position of the climatological zonal wave 3 ridges.
\citet{turner2017}, on the other hand, used the fourier amplitude of the
wave 3. To make these indices more intercomparable between themselves
and our data, we replicate them with slight diferences. We compute
\citet{raphael2004} baed on monthly means instead of three-monthly means
and we compute \citet{turner2017}'s fourier amplitude at 50S instead of
using the mean geopotential height between 55S and 65S.

\subsection{Data}

We use monthly geopotential height fields from the NCEP/NCAR Reanalysis
\citep{kalnay1996} for the period 1948 to 2017 and compute one
quasi-stationary wave for the whole period for each month, level and
wavenumber. Amplitude and phase for each wavenumber was estimated by
fitting a fourier transform to the mean geopotential height between
45\textbackslash degree and 65\textbackslash degree South and North for
each level and monthly record. For comparison, we also analyzed data
from ERA-Interim \citep{dee2011} and ERA-20C \citep{poli2016}.

We analyzed the data using the statistical programming language R
\citep{R-base}, using data.table \citep{R-data.table} and metR
\citep{R-metR} packages to read and transform it and ggplot2 package
\citep{R-ggplot2} to make the plots. The source code is available as
Figshare repository \citep{Campitelli2019-figshare}.

\section{Results}

\begin{figure}[h]

{\centering \subfloat[Northern Hemisphere.\label{fig:rao1}]{\includegraphics{fig/QS-ZW/rao-1} }\newline\subfloat[Southern Hemisphere.\label{fig:rao2}]{\includegraphics{fig/QS-ZW/rao-2} }

}

\caption{Seasonal cycle of amplitude of the geopotential height planetary waves 1, 2 and 3 (top, middle and bottom rows, respectively) computed as the amplitude of the mean wave ($A_\mathrm{QSk}$, left column) and the mean amplitude of the monthly waves ($\overline{A_\mathrm{ZW} }$, right column) in meters.}\label{fig:rao}
\end{figure}

Figure \ref{fig:rao} shows the seasonal cycle of the amplitude of
planetary waves N based on monthly mean geopotential height fields
computed between 1948 and 2017. We computed the left column
(\(A_\mathrm{QS}\)) as the amplitude of the average geopotential height
field for each month, level and wavenumber, and the right column
(\(\overline{A_\mathrm{ZW}}\)) as the average amplitude of the 70
individual fields.

Figure \ref{fig:rao1} shows that in the Northern Hemisphere
\(A_\mathrm{QS}\) and \(\overline{A_\mathrm{ZW}}\) have a similar
seasonal cycle with similar vertical extent for the three wavenumbers.
In the Southern Hemisphere this is true only for wavenumber 1 (Figure
\ref{fig:rao2}). However, \(A_\mathrm{QS2}\) has much lower values than
\(\overline{A_\mathrm{ZW2}}\) and its seasonal cycle is less defined.
Moreover, \(A_\mathrm{QS3}\) has a smaller magnitude than
\(\overline{A_\mathrm{ZW3}}\) and even though their overall structure is
similar (one relative maximum in February-March in the middle
troposphere and another in July-August that extends to the lower
stratosphere), they differ in the details. \(A_\mathrm{QS3}\) has a
local minimum in November that is absent in
\(\overline{A_\mathrm{ZW3}}\). The relative contribution of each
wavenumber is also different. While \(\overline{A_\mathrm{ZW2}}\)
dominates over \(\overline{A_\mathrm{ZW3}}\) in the stratosphere and is
of similar magnitude in the troposphere, \(A_\mathrm{QS3}\) dominates
over \(A_\mathrm{QS2}\) throughout the year and in every level except in
the aforementioned November minimum.

\begin{figure}[h]

{\centering \includegraphics{fig/QS-ZW/stationarity-1} 

}

\caption{Seasonal cycle of stationarity of the geopotential height planetary waves 1, 2 and 3 (top, middle and bottom rows, respectively) at 50\degree N and 50\degree S (left and right columns, respectively) computed using Equation \ref{eq:S}. The black line marks $\hat{S} = 0.4$ for reference.}\label{fig:stationarity}
\end{figure}

The differences between \(A_\mathrm{QS}\) and
\(\overline{A_\mathrm{ZW}}\) are quantified in Figure
\ref{fig:stationarity}, which shows \(\hat{S}\) for wavenumbers 1 to 3
computed using Equation \ref{eq:S}.

In the Nothern Hemisphere, planetary waves 1, 2 and 3 are highly
stationary in almost every month and level, and even more so planetary
wave 1 in the Southern Hemisphere.

In the SH, planetary wave 2 stationarity has a semiannual cycle. It
reaches its maximum in April and in August-September, decreasing to a
deep minimum centered in June. On the other hand, planetary wave 3
stationarity peaks in February and slowly decreases towards a November
deep minimum after which increases sharply.

\begin{figure}[h]

{\centering \includegraphics{fig/QS-ZW/s-timeseries-1} 

}

\caption{Stationarity for wavenumbers 1 to 3 at 50\degree N and  50\degree S (top and bottom panels, respectively) at 500hPa.}\label{fig:s-timeseries}
\end{figure}

As we computed \(\hat{S}\) using the whole period for Figure
\ref{fig:stationarity}, it represents the mean stationarity between 1948
and 2017. So, to analyse stationarity chances over time, we computed
\(\hat{S}\) using 10-year overlapping intervals for each wavenumber at
both studied latitudes (Figure \ref{fig:s-timeseries}). Planetary wave
stationarity remained high and constant for wavenumbers 1 to 3 in the
Northern Hemisphere and 1 in the Southern Hemisphere but wavenumbers 2
and 3 in the Southern Hemisphere show interdecadal variations. Planetary
wave 2 stationarity oscillated around 0.3 with maximums in the 50's,
70's and 00's. Planetary wave 3 stationarity jumped from zero to more
than 0.5 in less than five years in the 50's and then oscillated around
0.6 with a strong maximum in the late 80's.

The study was also made using ERA Interim reanalysis and the results for
the overlapping period are similar (not shown). The latter lends some
credence the observed decadal variations not being an artifact of the
reanalysis model. However, the sudden shift in planetary wave 3
stationarity observed between 1950 and 1960 is probably spurious and we
couldn't find it using data from ERA 20C (not shown).

\subsection{Comparison with other indices}

(This needs to be integrated)

\begin{figure}[h]

{\centering \includegraphics{fig/QS-ZW/index-comparison-1} 

}

\caption{Raphael (2003) and Turner et.al. indices compared with the amplitude of the zonal wave projected onto the climatological zonal wave. All of them computed at 500hPa and 50\degree S.}\label{fig:index-comparison}
\end{figure}

Figure \ref{fig:index-comparison} shows the relationship between the
amplitude of the projection of the zonal wave 3 onto the climatological
wave 3 and two zonal wave indeces propsed in the literature.
\citet{raphael2004}'s index is highly correlated with ours (correlation
= 0.73) and the relationship is quite linear in nature. Negative values
of \citet{raphael2004}'s index are generally associated to zonal waves
with negative projection onto the climatological mean. The main
difference between \citet{raphael2004}'s index is that it consideres
temporal anomalies instead of zonal anomalies. Computing the index using
zonal anomalies instead increases the correlation to 0.95.

The relationship with \citet{turner2017}'s approach is very different.
While the correlation is not low (0.59), but the relationship is far
from linear. Mainly, \citet{turner2017}'s index fails to capture the
fact that in a considerable number of dates, zonal wave 3 has a big
amplitude but with negative centers where climatologically there are
positive centers. Since this leads to oposite patterns of circulation,
corrlations derived from this index are not physically meaningful.

\section{Conclusions}

We assessed the stationarity levels of planetary waves at both
hemispheres using different quantitative measures. We confirmed that
planetary wave stationarity associated with SH planetary wave 1 is high
and constant throughout the year and period. Instead stationarity levels
for both waves 2 and 3 vary on intraseasonal and interdecadal
timescales. On the other hand, as it was described in the literature,
planetary wave stationarity in the Northern Hemisphere is higher and
varies much less.

Planetary waves can be both forced by the surface and excited by
internal variability. Assuming that the later process will not result in
a phase preference, higher stationarity would be evidence of stronger
forcing or, more strictly, stronger response. In the Northern
Hemisphere, topography and thermal contrasts are the main forcings of
planetary waves \citep{chen1988}, which explains their highly and not
variable stationary nature. In the SH, only planetary wave 1 seems to be
the result of mainly surface forcings. Planetary waves 2 and 3 seem to
be a product of internal variability and surface forcing components in
similar magnitude.

Since in the SH the amplitude of the mean planetary wave can differ
greatly from the mean amplitude of planetary waves, care must be taken
when interpreting the literature. Some studies analyze the former
\citep[e.g.~][\citet{quintanar1995a}, \citet{raphael2004}]{vanloon1972}
while others analyze the later \citep[e.g.~][\citet{turner2017},
\citet{irving2015}]{rao2004}. For instance, \citet{irving2015} compare
their planetary wave activity index with \citet{raphael2004}'s wave 3
index and conclude that the later cannot account for events with waves
far removed from their climatological position. However, being an index
of the zonal wave component in phase with the quasi-stationary wave,
this is by design.

The explorations of both zonal waves and quasi-stationary waves can lead
to novel levels of analysis. For example, \citet{smith2012} used their
phase relationship to show that linear interference between the zonal
waves 1 and quasi-stationary wave 1 was related to vertical wave
activity transport at the tropopause. Here, we showed it can be used to
define a metric of stationarity of quasi-stationary waves, but other
applications are also possible.

A deeper knowledge of interseasonal and decadal variations in planetary
wave stationarity could help to better understand the variability in the
responses of mid-latitudes atmospheric circulation to surface forcing
caused either by the strength of the forcing or the sensitivity of the
atmosphere to the forcing.

\acknowledgments

A version-controlled repository of the code that generated this article
can be found at http://github.com/eliocamp/qs-zw. And a snapshot of said
repository can be found at https://figshare.com/s/e72154e67b0cd8cc1045.

Funding statement:

\begin{itemize}
\item
  ``Climate Services Through Knowledge Co-Production: A Euro-South
  American Initiative For Strengthening Societal Adaptation Response to
  Extreme Events (CLIMAX)'', Belmont Forum/ANR-15-JCL/-0002-01. France.
\item
  ``Interacciones entre patrones climáticos de gran escala y su impacto
  en el sur de Sudamérica''. UBACYT 20020170100428BA. University of
  Buenos Aires, Argentina.
\end{itemize}

\bibliography{qszw.bib,qszw-pkgs.bib}


\end{document}
